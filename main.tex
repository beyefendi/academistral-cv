%-------------------------
% Academistrial Resume Template
% Author: Emre Süren
% Credit: Inspired by many CV templates on GitHub
%------------------------

\documentclass[letterpaper,11pt]{article}

\usepackage{latexsym}
\usepackage[empty]{fullpage}
\usepackage{titlesec}
\usepackage{marvosym}
\usepackage[usenames,dvipsnames]{color}
\usepackage{verbatim}
\usepackage{enumitem}
\usepackage[hidelinks]{hyperref}
\usepackage{fancyhdr}
\usepackage[english]{babel}
\usepackage{tabularx}
\usepackage{hyphenat}
\usepackage{fontawesome}
\input{glyphtounicode}

\usepackage{soul} % Text underline while wrapping text
\usepackage{ifthen}

%---------- FONT OPTIONS ----------
% sans-serif
% \usepackage[sfdefault]{FiraSans}
% \usepackage[sfdefault]{roboto}
% \usepackage[sfdefault]{noto-sans}
% \usepackage[default]{sourcesanspro}

% serif
% \usepackage{CormorantGaramond}
% \usepackage{charter}


%---------- PAGE LAYOUT ----------
\pagestyle{fancy}
\fancyhf{} % clear all header and footer fields
\fancyfoot{}
\renewcommand{\headrulewidth}{0pt}
\renewcommand{\footrulewidth}{0pt}

% Adjust margins
\addtolength{\oddsidemargin}{-0.5in}
\addtolength{\evensidemargin}{-0.5in}
\addtolength{\textwidth}{1in}
\addtolength{\topmargin}{-.5in}
\addtolength{\textheight}{1.0in}

\urlstyle{same}

\raggedbottom
\raggedright
\setlength{\tabcolsep}{0in}

% Sections formatting
\titleformat{\section}{
    \vspace{-4pt}\scshape\raggedright\large}
    {}{0em}{}[\color{black}\titlerule \vspace{-2pt}]

% Ensure that generate pdf is machine readable/ATS parsable
\pdfgentounicode=1

%---------- CUSTOM COMMANDS FOR SECTIONS ----------

% #1: Title #2: Year
\newcommand{\titleYear}[2]{
    % \vspace{-2pt}
    \item
    \begin{tabular*}{0.97\linewidth}{l@{\extracolsep{\fill}}r}
      #1 & \textit{\small #2}
    \end{tabular*}%\vspace{-7pt}
}
% #1: Title #2: Link #3: Year 
\newcommand{\titleLinkYear}[3]{
    % \vspace{-2pt}
    \item
    \begin{tabular*}{0.97\linewidth}[t]{l@{\extracolsep{\fill}}r}
        #1 \small{\href{#2}{\faLink}} & \textit{\small #3}
    \end{tabular*}%\vspace{-7pt}
}

% #1: Title #2: Year #3: Organization
\newcommand{\titleYearOrg}[3]{
    %\vspace{-2pt}
    \item
    \begin{tabular*}{0.97\linewidth}[t]{l@{\extracolsep{\fill}}r}
        #1 & \textit{\small #2} \\
        \textit{\small#3}
    \end{tabular*}%\vspace{-7pt}
}

% #1: Title #2: Year #3: Organization #4: Place
\newcommand{\titleYearOrgPlace}[4]{
    %\vspace{-2pt}
    \item
    \begin{tabular*}{0.97\linewidth}[t]{l@{\extracolsep{\fill}}r}
        #1 & \textit{\small #2} \\
        \textit{\small#3} & \textit{\small #4}
    \end{tabular*}%\vspace{-7pt}
}

% #1: Title #2: Link #3: Year #4: Role
\newcommand{\titleLinkYearRole}[4]{
    %\vspace{-2pt}
    \item
    \begin{tabular*}{0.97\linewidth}[t]{l@{\extracolsep{\fill}}r}
        #1 \small{\href{#2}{\faLink}} & \textit{\small #3} \\
        \textit{\small#4}
    \end{tabular*}%\vspace{-7pt}
}

% #1: (Bold) Title #2: Year #3: Organization #4: Link #5: Place
\newcommand{\bTitleYearOrgLinkPlace}[5]{
    %\vspace{-2pt}
    \item
    \begin{tabular*}{0.97\textwidth}[t]{l@{\extracolsep{\fill}}r}
        \textbf{#1} & \textit{\small #2} \\
        \textit{\small#3} \small{\href{#4}{\faLink}} & \small \faMapMarker \hspace{.5pt} #5
    \end{tabular*}%\vspace{0pt}
}

% #1: Authors #2: Paper #3: Journal #4: Pages #5: Year
\newcommand{\publicationsItem}[5]{
    % \vspace{-2pt}
    \item
    {#1 \textit{#2} \ul{#3}#4\textit{#5}}
    % \vspace{7pt}
}

% #1: Title #2: Year #3: Agency #4: Amount  # 7:Role #8: Objective @TODO
\newcommand{\fundingItem}[9]{
    %\vspace{-2pt}
    \item
    \begin{tabular*}{0.97\textwidth}[t]{p{0.85\linewidth}@{\extracolsep{\fill}}r}
        \textbf{#1} & \textit{\small #2}  \\
    \end{tabular*}
    {\setlength{\tabcolsep}{1em}
    \begin{tabular*}{0.97\textwidth}[t]{ll}
        Agency & #3 \small{\href{#4}{\faLink}} \\
        Fund & #6 $|$ #7\\
        PIs & #5 \\
        Role & #8 \\
        Objective & #9 \\
    \end{tabular*}}%\vspace{0pt}
}

% #1: Title #2: Year
\newcommand{\teachingItem}[2]{
    %\vspace{-2pt}
    \item
    \begin{tabular*}{0.97\linewidth}[t]{l@{\extracolsep{\fill}}r}
        #1 & \textit{\small #2}
    \end{tabular*}%\vspace{-7pt}
}

% #1: Name #2: Title #3: E-mail #4: Organization 
\newcommand{\referenceItem}[4]{
    \vspace{-2pt}
    \begin{tabular}{0.97\textwidth}
        #1, \emph{#2} \small{\href{mailto:#3}{\faEnvelope}} \\
        \small #4
    \end{tabular}\vspace{0pt}
    \quad
}

% Takes two parameters. One is mandatory and one is optional
% Optional is the first one and its value is <empty>
% #1 is Label type and #2 is Sub Title
% Optional parameter is passed by using []
% So the final paramaters will be like that []{}
% Actually both parameters are optional but newcommand can have onle one optional parameter
\newcommand{\itemizeCVBegin}[2][]{

    % Sub Title is empty
    \ifthenelse{\equal{#2}{}}
    {
    % Sub Title is set
    }{
        \textbf{#2}
    }

    % Itemize without label
    \ifthenelse{\equal{#1}{nolabel}}
    {
        \begin{itemize}[leftmargin=0.15in, label={}]
    % Itemizes with default label 
    }{
        \begin{itemize}
    }
}
\newcommand{\itemizeCVEnd}{\end{itemize}}


% \usepackage{xparse}
% \NewDocumentCommand{\itemizeCVBegin}{
%   O{} % Title argument is Optional, delimited by [], if not given use NULL
%   D(){} % Label argument is Optional, delimited by (), if not given use NULL
%   }{%    
    % \ifthenelse{\equal{#1}{}}
    % {
    % }{
    %     \textbf{#1}
    % }

    % \ifthenelse{\equal{#2}{nolabel}}
    % {
        % \begin{itemize}[leftmargin=0.15in, label={}]
    % }{
    %     \begin{itemize}
    % }
    % }



% Default item
\newcommand{\resumeItem}[1]{
  \item\small{{#1 \vspace{-2pt}}}
}
% Default item [No use delete]
\newcommand{\resumeSubItem}[1]{\resumeItem{#1}\vspace{-4pt}}
\renewcommand\labelitemii{$\vcenter{\hbox{\tiny$\bullet$}}$}

% Latex bullet symbol: \item[\ding{94}]

% Select Mode
\newcommand{\mymode}{academic}
% \newcommand{\mym≥ode}{industry}

%-------------------------------------------
%%%%%%%%%%%%%%  RESUME START  %%%%%%%%%%%%%%
\begin{document}

\ifthenelse{\equal{\mymode}{academic}}
{% True case
    %---------- HEADING ----------

% OLD: Personal page: https://www.kth.se/profile/emsuren/

\begin{center}

    \textbf{\Huge \scshape Emre Süren} \\ \vspace{3pt}
    \small
    % For privacy reasons remove it
    % \faMobile           \hspace{.5pt} \href{tel:+460730299435}{+46 073 029 94 35}
    % $|$
    \faEnvelope         \hspace{.5pt} \href{mailto:emresuren@gmail.com}{E-mail}
    $|$
    \faLinkedinSquare   \hspace{.5pt} \href{https://www.linkedin.com/in/emre-süren-6b1898a6}{LinkedIn}
    $|$
    \faGithub           \hspace{.5pt} \href{https://github.com/beyefendi}{GitHub}
    $|$
    \faGraduationCap    \hspace{.5pt} \href{https://scholar.google.com/citations?user=CnQIGjQAAAAJ&hl=en}{Scholar}
    $|$
    \faGlobe            \hspace{.5pt} \href{https://people.kth.se/~emsuren}{Portfolio}
    $|$
    \faMapMarker        \hspace{.5pt} \href{https://goo.gl/maps/rwe7HaYiAAsTY1GF8}{Stockhom, Sweden}

\end{center}

    %----------- RESEARCH INTERESTS -----------

\section{Research Interests}

    \itemizeCVBegin[nolabel]{}

        \small
        \item{Vulnerability discovery, Exploit development, Digital forensics, IoT, AI-based cybersecurity applications, open-source, and Capture the Flag (CTF).}

    \itemizeCVEnd


    %----------- EDUCATION -----------

\section{Education}


    \itemizeCVBegin[nolabel]{}
    
        \bTitleYearOrgLinkPlace
            {Ph.D. in Information Systems (Cybersecurity)}
            {2014 \textbf{--} 2019}
            {Middle East Technical University}
            {https://www.metu.edu.tr/}
            {Ankara, Turkey}
            \small A novel and efficient detection technique for zero-day web-based exploit kit families
            
        \bTitleYearOrgLinkPlace
            {Master’s in Information Systems (Cybersecurity)}
            {2014 \textbf{--} 2019}
            {Middle East Technical University}
            {https://www.metu.edu.tr/}
            {Ankara, Turkey}
            \small Detection of malicious web pages

        \bTitleYearOrgLinkPlace
            {Bachelor in Computer Engineering}
            {2012 \textbf{--} 2014}
            {Hacettepe University}
            {https://www.hacettepe.edu.tr/english/}
            {Ankara, Turkey}
            
    \itemizeCVEnd

    %----------- WORK EXPERIENCE -----------

% @TODO
% Develop an \optional tag
% Replace with cvItemListStart
% If the industry mode is on print the given text otherwise no

\section{Employment}

  
    \itemizeCVBegin[nolabel]{}
    
        \educationItem
            {Postdoctoral Researcher}
            {https://www.kth.se/en}
            {2020 \textbf{--} 2023}
            {KTH Royal Institute of Technology}
            {Stockholm}
            \small School of Electrical Engineering and Computer Science
            % \resumeItemListStart
            %     \resumeItem{IoT Vulnerability research and exploit development}
            % \resumeItemListEnd
            
        \educationItem
            {Principal Lecturer}
            {No link}
            {2017 \textbf{--} 2020}
            {Hacker School \& Freelance}
            {Ankara}
            \small
            % {}\vspace{-15pt} % SPACE
            % \resumeItemListStart
            %     \resumeItem{Cybersecurity curriculum development, course content development, teaching, organizing Capture the Flag (CTF) competitions, National Cyber Shield Exercise 2017, and delivering awareness briefings.}
            % \resumeItemListEnd

        \educationItem
            {Senior Engineer (Incident Response \& Threat Intelligence Analyst)}
            {https://www.havelsan.com.tr/en}
            {2013 \textbf{--} 2017}
            {HAVELSAN}
            {Ankara}
            \small Turkish Air Force Cyber Defense Command
            % \resumeItemListStart
            %     \resumeItem{Military incident response (Military CERT/HvKK-CERT), cyber threat intelligence, malware analysis, Turkish Air Force Cyber Shield Exercise (2013, 2015), delivering briefings on intelligence, discovery, and awareness to the generals and arm/force commanders, and organizing Capture the Flag competitions.}
            % \resumeItemListEnd

        \educationItem
            {Researcher (Penetration Tester)}
            {https://tubitak.gov.tr/en}
            {2007 \textbf{--} 2013}
            {TUBITAK}
            {Ankara}
            \small National Cybersecurity Institute
            % \resumeItemListStart
            %     \resumeItem{Web applications security testing, offensive source code analysis, penetration testing, National Incident Response (National CERT/TR-CERT), National Cyber Shield Exercise 2013, teaching the web app security testing course.}
            % \resumeItemListEnd

    \itemizeCVEnd
    %----------- PUBLICATIONS -----------

% @TODO
% Counter for paper index
% Automatically make bold Author name
% Set commas of the page numbers in the command definition

\section{Publications}


    \itemizeCVBegin[nolabel]{Journals}

        \publicationsItem
            {[5] F. Heiding, \textbf{E. Suren}, J. Olegård, R. Lagerström.}
            {Penetration testing of connected households.}
            {Computers \& Security}
            {, 126:103067, }
            {February 2023}

        \publicationsItem
            {[4] \textbf{E. Suren}, F. Heiding, J. Olegård, R. Lagerström.} 
            {PatrIoT: Practical and agile threat research for IoT.} 
            {International Journal of Information Security (IJIS'22)}
            {, 213–233, }
            {December 2022}

        \publicationsItem
            {[3] \textbf{E. Suren}.} 
            {ZEKI: Unsupervised Zero-day Exploit Kit Intelligence.} 
            {Turkish Journal of Electrical Engineering \& Computer Sciences (TJEECS'20)}
            {, 28(5):1859-1870, } 
            {March 2020}

        \publicationsItem
            {[2] \textbf{E. Suren}, P. Angin, N. Baykal.} 
            {I see EK: A Lightweight Technique to Reveal Exploit Kit family by Overall URL Patterns of Infection Chains.} 
            {Turkish Journal of Electrical Engineering \& Computer Sciences (TJEECS'19)}
            {, 27(5):3713-3728, }
            {September 2019}

        \publicationsItem
            {[1] \textbf{E. Suren} and P. Angin.}
            {Know Your EK: A Content and Workflow Analysis Approach for Exploit Kits.}
            {Journal of Internet Services and Information Security (JISIS'19)}
            { , 9(1):24–47, }
            {February 2019}

    \itemizeCVEnd


    \itemizeCVBegin[nolabel]{Conference Proceedings}

        \publicationsItem
            {[1] Ciklabakkal, Ege and Donmez, Ataberk and Erdemir, Mert and \textbf{Suren, Emre} and Yilmaz, Mert Kaan and Angin, Pelin.} 
            {ARTEMIS: An intrusion detection system for MQTT attacks in Internet of Things.}
            {38th Symposium on Reliable Distributed Systems (SRDS'19)}
            {, 369-3692, }
            {October 2019}
        
    \itemizeCVEnd
    %----------- FUNDING -----------

% @TODO
% Fix the \hbox error

\section{Funding}


    \itemizeCVBegin[nolabel]{}
        
        \fundingItem
            {DFIRent - Next-generation AI-assisted Digital Forensics Tools}
            {2024 \textbf{--} 2029}
            {European Union - DIGITAL Europe}
            {https://digital-strategy.ec.europa.eu/en/news/eu13-billion-digital-europe-programme-europes-digital-transition-and-cybersecurity}
            {Emre Süren, to be determined}
            {6M € in total / 1.5M € our share}
            {In preparation}
            {Task leader}
            {AI-based digital forensics on current operating systems, memory, and IoT devices}

        \fundingItem
            {AI\textsuperscript{2}IoT – AI-based IoT Vulnerability Exploitation}
            {2024 \textbf{--} 2029}
            {European Research Council - Starting Grant}
            {https://erc.europa.eu/apply-grant/starting-grant}
            {Emre Süren}
            {1.5M €}
            {In preparation}
            {Project leader}
            {Developing AI-based methods and tools for IoT vulnerability discovery}

        \fundingItem
            {IoTVR – Vulnerability Research on Consumer IoT Devices}
            {2024 \textbf{--} 2026}
            {Swedish Research Council – Cyber and Information Security}
            {https://www.vr.se/english/applying-for-funding/calls/2022-11-10-project-grant-for-research-into-cyber-and-information-security.html}
            {Emre Süren}
            {1.4M SEK ($\sim$0.14K €)}
            {Pending}
            {Project leader}
            {Vulnerability research on current IoT hardware, firmware, and radio protocols}

        \fundingItem
            {EnergyShield – Integrated Cybersecurity Solution for the Vulnerability Assessment, Monitoring, and Protection of Critical Energy Infrastructures}
            {2019 \textbf{--} 2022}
            {European Union – Horizon 2020}
            {https://energy-shield.eu/}
            {Robert Lagerström (PostDoc supervisor)}
            {8M € in total / $\sim$0.6M € our share}
            {EU funding}
            {Senior researcher \& Task leader}
            {Vulnerability research on European energy utilities}

        \fundingItem
            {VASA – Viable cities and Attack Simulation \& threAt modeling}
            {2020 \textbf{--} 2022}
            {Swedish Energy Agency – Viable Cities Program}
            {https://en.viablecities.se/foi-projekt/hotmodellering-och-attacksimulering-av-livskraftiga-stader/}
            {Robert Lagerström (PostDoc supervisor)}
            {6M SEK ($\sim$0.6M €)}
            {National funding}
            {Senior researcher \& Project leader}
            {Vulnerability research on smart building devices}
                        
    \itemizeCVEnd
    %----------- TEACHING -----------

\section{Teaching}

    
    \itemizeCVBegin{KTH Royal Institute of Technology}
    
        \teachingItem
            {[EP2780 - Graduate] Digital Forensics and Incident Response \small{\href{https://www.kth.se/student/kurser/kurs/EP2780?l=en}{\faLink}}}
            {2022}
            
    \itemizeCVEnd


    \itemizeCVBegin{Hacker School \& Freelance}

        \teachingItem
            {Digital Forensics and Incident Response}
            {2017 \textbf{--} 2020}
        
        \teachingItem
            {Web Application Hacking}
            {2017 \textbf{--} 2020}
           
        \teachingItem
            {Ethical Hacking (Penetration Testing)}
            {2017 \textbf{--} 2020}
        
    \itemizeCVEnd


    %----------- SUPERVISING -----------

% @TODO
% Pass the link in a better way 

\section{Supervising}

  
    \itemizeCVBegin[nolabel]{Graduate (\small{Diva profile \href{https://www.diva-portal.org/smash/resultList.jsf?dswid=-4097&query=emre+süren}{\faLink}})}

        \cvItem
            {[7] Exploiting memory corruption bugs for remote code execution on IoT firmware}
            {}
            {2023 \textbf{--} 2023}
            {Erik Mickols, KTH}
            {}
        \cvItem
            {[6] Penetration testing of current smart thermostats}
            {}
            {2023 \textbf{--} 2023}
            {Adam Lindberg, KTH}
            {}
        \cvItem
            {[5] Nationwide password patterns of Swedish people}
            {https://www.diva-portal.org/smash/record.jsf?pid=diva2:1751652}
            {2022 \textbf{--} 2022}
            {Daniel Gustafsson, KTH}
            {}
        \cvItem
            {[4] Automated security analysis of firmware}
            {https://www.diva-portal.org/smash/record.jsf?pid=diva2:1704788}
            {2022 \textbf{--} 2022}
            {Farrokh Bolandi, KTH}
            {}
        \cvItem
            {[3] (Classified) Hacking the airport X-ray machine}
            {https://SecretProject.nda}
            {2022 \textbf{--} 2022}
            {Felix Zuber, KTH}
            {}
        \cvItem
            {[2] Threat modeling and penetration testing of an IoT product. Yanzi IoT network}
            {https://www.diva-portal.org/smash/record.jsf?pid=diva2:1636796}
            {2021 \textbf{--} 2021}
            {Diyala Isabar, KTH}
            {}
        \cvItem
            {[1] Security analysis of a Beckhoff CX-9020 programmable logic controller}
            {https://www.diva-portal.org/smash/record.jsf?pid=diva2:1603740}
            {2021 \textbf{--} 2021} 
            {Liam Carter, KTH}
            {}
    
    \itemizeCVEnd


    \itemizeCVBegin[nolabel]{Undergraduate}
        
        \cvItem
            {[3] Penetration testing of ten popular Swedish Android applications}
            {https://www.diva-portal.org/smash/record.jsf?pid=diva2:1704891}
            {2022 \textbf{--} 2022}
            {Alexander Astély \& Johan Ekroth, KTH}
            {}
        \cvItem
            {[2] Security evaluation of ten Swedish mobile applications}
            {https://www.diva-portal.org/smash/record.jsf?pid=diva2:1701868}
            {2022 \textbf{--} 2022}
            {Jens Ekenblad \& Stefan Andres Garrido Valenzuela, KTH}
            {}
        \cvItem
            {[1] Trends and scientometrics in cyber security research}
            {https://www.diva-portal.org/smash/record.jsf?pid=diva2:1707874}
            {2021 \textbf{--} 2022}
            {Jesper Slagarp \& Elvira Häggström, KTH}
            {}
    
    \itemizeCVEnd
    
    %----------- ACADEMIC SERVICE -----------

\section{Academic Service}
    
    
    \itemizeCVBegin{Technical Program Committee} 
        
        \titleLinkYearRole
            {International Symposium on Networks, Computers and Communications (ICNCC)}
            {https://www.isncc-conf.org/}
            {2023}
            {Member}
        \titleLinkYearRole
            {DFIR Review (DFIRR)}
            {https://dfir.pubpub.org/}
            {2023}
            {Member}
        \titleLinkYearRole
            {ADFSL Conference on Digital Forensics, Security and Law}
            {https://www.digitalforensics-conference.org/committees}
            {2022 \textbf{--} 2023}
            {Member}
        \titleLinkYearRole
            {Swedish National Computer Networking Workshop (SNCNW)}
            {http://sncnw.se/2022/org.html}
            {2022}
            {Co-chair}
    
    \itemizeCVEnd

    
    \itemizeCVBegin{Reviewer}
        
        \titleLinkYear
            {IEEE Security \& Privacy (S\&P)}
            {https://ieeexplore.ieee.org/xpl/RecentIssue.jsp?punumber=8013}
            {2023}
        \titleLinkYear
            {Information Security Journal: A global perspective (ISJGP)}
            {https://www.tandfonline.com/toc/uiss20/current}
            {2023}
        \titleLinkYear
            {Computer Science Review (CSR)}
            {https://www.journals.elsevier.com/computer-science-review}
            {2022 \textbf{--} 2023}
        \titleLinkYear
            {International Journal of Critical Infrastructure Protection (IJCIP)}
            {https://www.journals.elsevier.com/international-journal-of-critical-infrastructure-protection}
            {2022}
        \titleLinkYear
            {Business \& Information Systems Engineering (BISE)}
            {https://www.springer.com/journal/12599}
            {2022}
        \titleLinkYear
            {Practice of Enterprise Modelling Conference (PoEM)}
            {https://dblp.org/db/conf/ifip8-1/poem2021.html}
            {2022}
        \titleLinkYear
            {IEEE Int. Enterprise Distributed Object Computing Conference (EDOC)}
            {http://edocconference.org/?i=1}
            {2022}
        \titleLinkYear
            {IEEE Wireless Communications and Networking Conference (WCNC)}
            {https://wcnc2023.ieee-wcnc.org/}
            {2022}
        
    \itemizeCVEnd


    \itemizeCVBegin{Administrative Service \& Membership}
        
        \titleLinkYearRole
            {European Union Agency for Cybersecurity (ENISA)}
            {https://www.enisa.europa.eu/procurement/cei-list-of-nis-experts}
            {2023 \textbf{--} current}
            {CEI Listed expert}
            % {EU}
        \titleLinkYearRole
            {KTH NSE Cybersecurity lab}
            {https://nse.digital/pages/contribute.html}
            {2020 \textbf{--} 2023}
            {Member \& Author}
            % {Stockholm}
        \titleLinkYearRole
            {Turkish Criminal Courts}
            {Confidential}
            {2014 \textbf{--} 2019}
            {Authorized digital forensics specialist}
            % {Ankara}
    
    \itemizeCVEnd
    %----------- INVITED TALKS & PANELS -----------

% NOTE: Links disappear quickly, no need, it is just overhead

\section{Invited Talks \& Panels}

  
    \itemizeCVBegin{}
    
        \titleYearOrg
            {CTF Competitions for Cybersecurity Education}
            {2022}
            {Education Track - KTH EECS Summer Event 2022}
            
    \itemizeCVEnd

    %----------- EVENTS ORGANIZED -----------

\section{Events Organized}


    \itemizeCVBegin{}
    
        \titleYearOrgPlace
            {EU Cyber Attaché Exercise}
            {2023}
            {KTH - CDIS Cyber range}
            {Stockholm}
        \titleYearOrgPlace
            {Turkish National Cybershield Exercise}
            {2017}
            {Hacker School}
            {Ankara}
        \titleYearOrgPlace
            {Turkish Military Cybershield Exercise}
            {2015}
            {HAVELSAN}
            {Ankara}
        \titleYearOrgPlace
            {Turkish National Cybershield Exercise}
            {2013}
            {TUBITAK}
            {Ankara}
    
    \itemizeCVEnd

    %----------- PERSONAL DEVELOPMENT -----------

% @TODO
% Pass the link in a better way 

\section{Personal Development}

    \itemizeCVBegin{Open-source security research tools}
    
        \cvItem
            {\textbf{Memotopsy.} Memory forensics tool.}
            {https://github.com/beyefendi/memotopsy}
            {2023}
            {Owner}
            {}
        \cvItem
            {\textbf{PatrIoT}. IoT vulnerability research methodology.}
            {https://github.com/beyefendi/penbook/tree/main/iot}
            {2021}
            {Owner}
            {}
        \cvItem
            {\textbf{Graudit}. Offensive source code review tool.}
            {https://github.com/wireghoul/graudit}
            {2020}
            {Contributor}
            {}
            
    \itemizeCVEnd

  
    \itemizeCVBegin{Private Training}
        
        \cvItem
            {Burp Suite Certified Practitioner}
            {}
            {2021}
            {PortSwigger}
            {Online}
        \cvItem
            {Advanced Web Attacks and Exploitation (OSWE)}
            {}
            {2021}
            {OffSec}
            {Online}
        \cvItem
            {Penetration Testing with Kali Linux (OSCP)}
            {}
            {2019}
            {OffSec}
            {Online}
        \cvItem
            {Reverse Engineering Malware}
            {}
            {2016}
            {SANS}
            {Las Vegas}
        \cvItem
            {Customized Malware Analysis}
            {}
            {2016}
            {Mandiant}
            {Ankara}
        \cvItem
            {Enterprise Incident Response}
            {}
            {2015}
            {Mandiant}
            {Ankara}
        \cvItem
            {Network Penetration Testing and Ethical Hacking}
            {}
            {2013}
            {SANS}
            {Bootcamp}
        \cvItem
            {Web App PenTesting and Ethical Hacking}
            {}
            {2012}
            {SANS}
            {Washington DC}        
        \cvItem
            {Security Essentials}
            {}
            {2011}
            {SANS}
            {Bootcamp}

    \itemizeCVEnd


    \itemizeCVBegin{Certificates (\small{{GIAC profile} \href{http://www.giac.org/certified-professional/emre-suren/128055}{\faLink} $|$ 
    {Acclaim profile} \href{https://www.youracclaim.com/user/emre-suren}{\faLink})}}

        \cvItem
            {Certified Ethical Hacker (CEH) v10}
            {}
            {2018}
            {}
            {}
        \cvItem
            {GIAC Reverse Engineering Malware (GREM)}
            {}
            {2016}
            {}
            {}
        \cvItem
            {TSE Certified Penetration Test Expert – Network \& System}
            {}
            {2015}
            {}
            {}
        \cvItem
            {TSE Certified Penetration Test Expert – Web \& Database}
            {}
            {2015}
            {}
            {}
        \cvItem
            {GIAC Penetration Tester Certification (GPEN)}
            {}
            {2013}
            {}
            {}
        \cvItem
            {GIAC Web Application Penetration Tester (GWAPT)}
            {}
            {2012}
            {}
            {}
        \cvItem
            {GIAC Security Essentials (GSEC)}
            {}
            {2011}
            {}
            {}
        \cvItem
            {Certified Ethical Hacker (CEH) v7}
            {}
            {2011}
            {}
            {}
  
    \itemizeCVEnd


    \itemizeCVBegin{Discovered vulnerabilities}

        \cvItem
            {TwinCAT OPC UA Server}
            {https://cert.vde.com/en/advisories/VDE-2021-051/}
            {2021}
            {CVE-2021-34594 - Unauthenticated any file manipulation through relative path traversal}
            {}

    \itemizeCVEnd


    \itemizeCVBegin{Exploit development}

        \cvItem
            {Ruckus IoT Controller (Ruckus vRIoT) 1.5.1.0.21}
            {https://www.exploit-db.com/exploits/49110}
            {2020}
            {Remote Code Execution}
            {}
            
    \itemizeCVEnd



}
{% False case
    \ifthenelse{\equal{\mymode}{industry}}
    {
        %---------- HEADING ----------

% OLD: Personal page: https://www.kth.se/profile/emsuren/

\begin{center}

    \textbf{\Huge \scshape Emre Süren} \\ \vspace{3pt}
    \small
    % For privacy reasons remove it
    % \faMobile           \hspace{.5pt} \href{tel:+460730299435}{+46 073 029 94 35}
    % $|$
    \faEnvelope         \hspace{.5pt} \href{mailto:emresuren@gmail.com}{E-mail}
    $|$
    \faLinkedinSquare   \hspace{.5pt} \href{https://www.linkedin.com/in/emre-süren-6b1898a6}{LinkedIn}
    $|$
    \faGithub           \hspace{.5pt} \href{https://github.com/beyefendi}{GitHub}
    $|$
    \faGraduationCap    \hspace{.5pt} \href{https://scholar.google.com/citations?user=CnQIGjQAAAAJ&hl=en}{Scholar}
    $|$
    \faGlobe            \hspace{.5pt} \href{https://people.kth.se/~emsuren}{Portfolio}
    $|$
    \faMapMarker        \hspace{.5pt} \href{https://goo.gl/maps/rwe7HaYiAAsTY1GF8}{Stockhom, Sweden}

\end{center}

        %----------- RESEARCH INTERESTS -----------

\section{Research Interests}

    \itemizeCVBegin[nolabel]{}

        \small
        \item{Vulnerability discovery, Exploit development, Digital forensics, IoT, AI-based cybersecurity applications, open-source, and Capture the Flag (CTF).}

    \itemizeCVEnd


        %----------- EDUCATION -----------

\section{Education}


    \itemizeCVBegin[nolabel]{}
    
        \bTitleYearOrgLinkPlace
            {Ph.D. in Information Systems (Cybersecurity)}
            {2014 \textbf{--} 2019}
            {Middle East Technical University}
            {https://www.metu.edu.tr/}
            {Ankara, Turkey}
            \small A novel and efficient detection technique for zero-day web-based exploit kit families
            
        \bTitleYearOrgLinkPlace
            {Master’s in Information Systems (Cybersecurity)}
            {2014 \textbf{--} 2019}
            {Middle East Technical University}
            {https://www.metu.edu.tr/}
            {Ankara, Turkey}
            \small Detection of malicious web pages

        \bTitleYearOrgLinkPlace
            {Bachelor in Computer Engineering}
            {2012 \textbf{--} 2014}
            {Hacettepe University}
            {https://www.hacettepe.edu.tr/english/}
            {Ankara, Turkey}
            
    \itemizeCVEnd

        %----------- WORK EXPERIENCE -----------

% @TODO
% Develop an \optional tag
% Replace with cvItemListStart
% If the industry mode is on print the given text otherwise no

\section{Employment}

  
    \itemizeCVBegin[nolabel]{}
    
        \educationItem
            {Postdoctoral Researcher}
            {https://www.kth.se/en}
            {2020 \textbf{--} 2023}
            {KTH Royal Institute of Technology}
            {Stockholm}
            \small School of Electrical Engineering and Computer Science
            % \resumeItemListStart
            %     \resumeItem{IoT Vulnerability research and exploit development}
            % \resumeItemListEnd
            
        \educationItem
            {Principal Lecturer}
            {No link}
            {2017 \textbf{--} 2020}
            {Hacker School \& Freelance}
            {Ankara}
            \small
            % {}\vspace{-15pt} % SPACE
            % \resumeItemListStart
            %     \resumeItem{Cybersecurity curriculum development, course content development, teaching, organizing Capture the Flag (CTF) competitions, National Cyber Shield Exercise 2017, and delivering awareness briefings.}
            % \resumeItemListEnd

        \educationItem
            {Senior Engineer (Incident Response \& Threat Intelligence Analyst)}
            {https://www.havelsan.com.tr/en}
            {2013 \textbf{--} 2017}
            {HAVELSAN}
            {Ankara}
            \small Turkish Air Force Cyber Defense Command
            % \resumeItemListStart
            %     \resumeItem{Military incident response (Military CERT/HvKK-CERT), cyber threat intelligence, malware analysis, Turkish Air Force Cyber Shield Exercise (2013, 2015), delivering briefings on intelligence, discovery, and awareness to the generals and arm/force commanders, and organizing Capture the Flag competitions.}
            % \resumeItemListEnd

        \educationItem
            {Researcher (Penetration Tester)}
            {https://tubitak.gov.tr/en}
            {2007 \textbf{--} 2013}
            {TUBITAK}
            {Ankara}
            \small National Cybersecurity Institute
            % \resumeItemListStart
            %     \resumeItem{Web applications security testing, offensive source code analysis, penetration testing, National Incident Response (National CERT/TR-CERT), National Cyber Shield Exercise 2013, teaching the web app security testing course.}
            % \resumeItemListEnd

    \itemizeCVEnd
        %----------- PUBLICATIONS -----------

% @TODO
% Counter for paper index
% Automatically make bold Author name
% Set commas of the page numbers in the command definition

\section{Publications}


    \itemizeCVBegin[nolabel]{Journals}

        \publicationsItem
            {[5] F. Heiding, \textbf{E. Suren}, J. Olegård, R. Lagerström.}
            {Penetration testing of connected households.}
            {Computers \& Security}
            {, 126:103067, }
            {February 2023}

        \publicationsItem
            {[4] \textbf{E. Suren}, F. Heiding, J. Olegård, R. Lagerström.} 
            {PatrIoT: Practical and agile threat research for IoT.} 
            {International Journal of Information Security (IJIS'22)}
            {, 213–233, }
            {December 2022}

        \publicationsItem
            {[3] \textbf{E. Suren}.} 
            {ZEKI: Unsupervised Zero-day Exploit Kit Intelligence.} 
            {Turkish Journal of Electrical Engineering \& Computer Sciences (TJEECS'20)}
            {, 28(5):1859-1870, } 
            {March 2020}

        \publicationsItem
            {[2] \textbf{E. Suren}, P. Angin, N. Baykal.} 
            {I see EK: A Lightweight Technique to Reveal Exploit Kit family by Overall URL Patterns of Infection Chains.} 
            {Turkish Journal of Electrical Engineering \& Computer Sciences (TJEECS'19)}
            {, 27(5):3713-3728, }
            {September 2019}

        \publicationsItem
            {[1] \textbf{E. Suren} and P. Angin.}
            {Know Your EK: A Content and Workflow Analysis Approach for Exploit Kits.}
            {Journal of Internet Services and Information Security (JISIS'19)}
            { , 9(1):24–47, }
            {February 2019}

    \itemizeCVEnd


    \itemizeCVBegin[nolabel]{Conference Proceedings}

        \publicationsItem
            {[1] Ciklabakkal, Ege and Donmez, Ataberk and Erdemir, Mert and \textbf{Suren, Emre} and Yilmaz, Mert Kaan and Angin, Pelin.} 
            {ARTEMIS: An intrusion detection system for MQTT attacks in Internet of Things.}
            {38th Symposium on Reliable Distributed Systems (SRDS'19)}
            {, 369-3692, }
            {October 2019}
        
    \itemizeCVEnd
        %----------- TEACHING -----------

\section{Teaching}

    
    \itemizeCVBegin{KTH Royal Institute of Technology}
    
        \teachingItem
            {[EP2780 - Graduate] Digital Forensics and Incident Response \small{\href{https://www.kth.se/student/kurser/kurs/EP2780?l=en}{\faLink}}}
            {2022}
            
    \itemizeCVEnd


    \itemizeCVBegin{Hacker School \& Freelance}

        \teachingItem
            {Digital Forensics and Incident Response}
            {2017 \textbf{--} 2020}
        
        \teachingItem
            {Web Application Hacking}
            {2017 \textbf{--} 2020}
           
        \teachingItem
            {Ethical Hacking (Penetration Testing)}
            {2017 \textbf{--} 2020}
        
    \itemizeCVEnd


        %----------- SUPERVISING -----------

% @TODO
% Pass the link in a better way 

\section{Supervising}

  
    \itemizeCVBegin[nolabel]{Graduate (\small{Diva profile \href{https://www.diva-portal.org/smash/resultList.jsf?dswid=-4097&query=emre+süren}{\faLink}})}

        \cvItem
            {[7] Exploiting memory corruption bugs for remote code execution on IoT firmware}
            {}
            {2023 \textbf{--} 2023}
            {Erik Mickols, KTH}
            {}
        \cvItem
            {[6] Penetration testing of current smart thermostats}
            {}
            {2023 \textbf{--} 2023}
            {Adam Lindberg, KTH}
            {}
        \cvItem
            {[5] Nationwide password patterns of Swedish people}
            {https://www.diva-portal.org/smash/record.jsf?pid=diva2:1751652}
            {2022 \textbf{--} 2022}
            {Daniel Gustafsson, KTH}
            {}
        \cvItem
            {[4] Automated security analysis of firmware}
            {https://www.diva-portal.org/smash/record.jsf?pid=diva2:1704788}
            {2022 \textbf{--} 2022}
            {Farrokh Bolandi, KTH}
            {}
        \cvItem
            {[3] (Classified) Hacking the airport X-ray machine}
            {https://SecretProject.nda}
            {2022 \textbf{--} 2022}
            {Felix Zuber, KTH}
            {}
        \cvItem
            {[2] Threat modeling and penetration testing of an IoT product. Yanzi IoT network}
            {https://www.diva-portal.org/smash/record.jsf?pid=diva2:1636796}
            {2021 \textbf{--} 2021}
            {Diyala Isabar, KTH}
            {}
        \cvItem
            {[1] Security analysis of a Beckhoff CX-9020 programmable logic controller}
            {https://www.diva-portal.org/smash/record.jsf?pid=diva2:1603740}
            {2021 \textbf{--} 2021} 
            {Liam Carter, KTH}
            {}
    
    \itemizeCVEnd


    \itemizeCVBegin[nolabel]{Undergraduate}
        
        \cvItem
            {[3] Penetration testing of ten popular Swedish Android applications}
            {https://www.diva-portal.org/smash/record.jsf?pid=diva2:1704891}
            {2022 \textbf{--} 2022}
            {Alexander Astély \& Johan Ekroth, KTH}
            {}
        \cvItem
            {[2] Security evaluation of ten Swedish mobile applications}
            {https://www.diva-portal.org/smash/record.jsf?pid=diva2:1701868}
            {2022 \textbf{--} 2022}
            {Jens Ekenblad \& Stefan Andres Garrido Valenzuela, KTH}
            {}
        \cvItem
            {[1] Trends and scientometrics in cyber security research}
            {https://www.diva-portal.org/smash/record.jsf?pid=diva2:1707874}
            {2021 \textbf{--} 2022}
            {Jesper Slagarp \& Elvira Häggström, KTH}
            {}
    
    \itemizeCVEnd
    
        %----------- ACADEMIC SERVICE -----------

\section{Academic Service}
    
    
    \itemizeCVBegin{Technical Program Committee} 
        
        \titleLinkYearRole
            {International Symposium on Networks, Computers and Communications (ICNCC)}
            {https://www.isncc-conf.org/}
            {2023}
            {Member}
        \titleLinkYearRole
            {DFIR Review (DFIRR)}
            {https://dfir.pubpub.org/}
            {2023}
            {Member}
        \titleLinkYearRole
            {ADFSL Conference on Digital Forensics, Security and Law}
            {https://www.digitalforensics-conference.org/committees}
            {2022 \textbf{--} 2023}
            {Member}
        \titleLinkYearRole
            {Swedish National Computer Networking Workshop (SNCNW)}
            {http://sncnw.se/2022/org.html}
            {2022}
            {Co-chair}
    
    \itemizeCVEnd

    
    \itemizeCVBegin{Reviewer}
        
        \titleLinkYear
            {IEEE Security \& Privacy (S\&P)}
            {https://ieeexplore.ieee.org/xpl/RecentIssue.jsp?punumber=8013}
            {2023}
        \titleLinkYear
            {Information Security Journal: A global perspective (ISJGP)}
            {https://www.tandfonline.com/toc/uiss20/current}
            {2023}
        \titleLinkYear
            {Computer Science Review (CSR)}
            {https://www.journals.elsevier.com/computer-science-review}
            {2022 \textbf{--} 2023}
        \titleLinkYear
            {International Journal of Critical Infrastructure Protection (IJCIP)}
            {https://www.journals.elsevier.com/international-journal-of-critical-infrastructure-protection}
            {2022}
        \titleLinkYear
            {Business \& Information Systems Engineering (BISE)}
            {https://www.springer.com/journal/12599}
            {2022}
        \titleLinkYear
            {Practice of Enterprise Modelling Conference (PoEM)}
            {https://dblp.org/db/conf/ifip8-1/poem2021.html}
            {2022}
        \titleLinkYear
            {IEEE Int. Enterprise Distributed Object Computing Conference (EDOC)}
            {http://edocconference.org/?i=1}
            {2022}
        \titleLinkYear
            {IEEE Wireless Communications and Networking Conference (WCNC)}
            {https://wcnc2023.ieee-wcnc.org/}
            {2022}
        
    \itemizeCVEnd


    \itemizeCVBegin{Administrative Service \& Membership}
        
        \titleLinkYearRole
            {European Union Agency for Cybersecurity (ENISA)}
            {https://www.enisa.europa.eu/procurement/cei-list-of-nis-experts}
            {2023 \textbf{--} current}
            {CEI Listed expert}
            % {EU}
        \titleLinkYearRole
            {KTH NSE Cybersecurity lab}
            {https://nse.digital/pages/contribute.html}
            {2020 \textbf{--} 2023}
            {Member \& Author}
            % {Stockholm}
        \titleLinkYearRole
            {Turkish Criminal Courts}
            {Confidential}
            {2014 \textbf{--} 2019}
            {Authorized digital forensics specialist}
            % {Ankara}
    
    \itemizeCVEnd
        %----------- INVITED TALKS & PANELS -----------

% NOTE: Links disappear quickly, no need, it is just overhead

\section{Invited Talks \& Panels}

  
    \itemizeCVBegin{}
    
        \titleYearOrg
            {CTF Competitions for Cybersecurity Education}
            {2022}
            {Education Track - KTH EECS Summer Event 2022}
            
    \itemizeCVEnd

        %----------- EVENTS ORGANIZED -----------

\section{Events Organized}


    \itemizeCVBegin{}
    
        \titleYearOrgPlace
            {EU Cyber Attaché Exercise}
            {2023}
            {KTH - CDIS Cyber range}
            {Stockholm}
        \titleYearOrgPlace
            {Turkish National Cybershield Exercise}
            {2017}
            {Hacker School}
            {Ankara}
        \titleYearOrgPlace
            {Turkish Military Cybershield Exercise}
            {2015}
            {HAVELSAN}
            {Ankara}
        \titleYearOrgPlace
            {Turkish National Cybershield Exercise}
            {2013}
            {TUBITAK}
            {Ankara}
    
    \itemizeCVEnd

        %----------- PERSONAL DEVELOPMENT -----------

% @TODO
% Pass the link in a better way 

\section{Personal Development}

    \itemizeCVBegin{Open-source security research tools}
    
        \cvItem
            {\textbf{Memotopsy.} Memory forensics tool.}
            {https://github.com/beyefendi/memotopsy}
            {2023}
            {Owner}
            {}
        \cvItem
            {\textbf{PatrIoT}. IoT vulnerability research methodology.}
            {https://github.com/beyefendi/penbook/tree/main/iot}
            {2021}
            {Owner}
            {}
        \cvItem
            {\textbf{Graudit}. Offensive source code review tool.}
            {https://github.com/wireghoul/graudit}
            {2020}
            {Contributor}
            {}
            
    \itemizeCVEnd

  
    \itemizeCVBegin{Private Training}
        
        \cvItem
            {Burp Suite Certified Practitioner}
            {}
            {2021}
            {PortSwigger}
            {Online}
        \cvItem
            {Advanced Web Attacks and Exploitation (OSWE)}
            {}
            {2021}
            {OffSec}
            {Online}
        \cvItem
            {Penetration Testing with Kali Linux (OSCP)}
            {}
            {2019}
            {OffSec}
            {Online}
        \cvItem
            {Reverse Engineering Malware}
            {}
            {2016}
            {SANS}
            {Las Vegas}
        \cvItem
            {Customized Malware Analysis}
            {}
            {2016}
            {Mandiant}
            {Ankara}
        \cvItem
            {Enterprise Incident Response}
            {}
            {2015}
            {Mandiant}
            {Ankara}
        \cvItem
            {Network Penetration Testing and Ethical Hacking}
            {}
            {2013}
            {SANS}
            {Bootcamp}
        \cvItem
            {Web App PenTesting and Ethical Hacking}
            {}
            {2012}
            {SANS}
            {Washington DC}        
        \cvItem
            {Security Essentials}
            {}
            {2011}
            {SANS}
            {Bootcamp}

    \itemizeCVEnd


    \itemizeCVBegin{Certificates (\small{{GIAC profile} \href{http://www.giac.org/certified-professional/emre-suren/128055}{\faLink} $|$ 
    {Acclaim profile} \href{https://www.youracclaim.com/user/emre-suren}{\faLink})}}

        \cvItem
            {Certified Ethical Hacker (CEH) v10}
            {}
            {2018}
            {}
            {}
        \cvItem
            {GIAC Reverse Engineering Malware (GREM)}
            {}
            {2016}
            {}
            {}
        \cvItem
            {TSE Certified Penetration Test Expert – Network \& System}
            {}
            {2015}
            {}
            {}
        \cvItem
            {TSE Certified Penetration Test Expert – Web \& Database}
            {}
            {2015}
            {}
            {}
        \cvItem
            {GIAC Penetration Tester Certification (GPEN)}
            {}
            {2013}
            {}
            {}
        \cvItem
            {GIAC Web Application Penetration Tester (GWAPT)}
            {}
            {2012}
            {}
            {}
        \cvItem
            {GIAC Security Essentials (GSEC)}
            {}
            {2011}
            {}
            {}
        \cvItem
            {Certified Ethical Hacker (CEH) v7}
            {}
            {2011}
            {}
            {}
  
    \itemizeCVEnd


    \itemizeCVBegin{Discovered vulnerabilities}

        \cvItem
            {TwinCAT OPC UA Server}
            {https://cert.vde.com/en/advisories/VDE-2021-051/}
            {2021}
            {CVE-2021-34594 - Unauthenticated any file manipulation through relative path traversal}
            {}

    \itemizeCVEnd


    \itemizeCVBegin{Exploit development}

        \cvItem
            {Ruckus IoT Controller (Ruckus vRIoT) 1.5.1.0.21}
            {https://www.exploit-db.com/exploits/49110}
            {2020}
            {Remote Code Execution}
            {}
            
    \itemizeCVEnd



    }
    {% False case
    }
}

%---A NEW SECTION DEVELOPMENT TEMPLATE ------

% \section{Section Name}
%   \vspace{3pt}
%   \resumeSectionListStart
%         \titleYearOrgCity
%             \resumeItemListStart
%                 \resumeItem{}
%             \small{ \item{ \textbf{ {} }}
%     \resumeSectionListEnd

%-------------------------------------------
\end{document}